\section{Basic Notions}
\subsection{Basic Definitions}
A \textbf{formal language} is a set of strings over finite sets of symbols.
We have different ways of describing such languages:
\begin{itemize}
  \item Finite automata
  \item Regular expressions
  \item Grammars
  \item Turing machines
\end{itemize}
\textit{Example:} The set of all strings of tokens generated by the grammar 
that describes the Java programming language is a formal language.\\

An \textbf{alphabet} is a finite set of symbols such as letters, digits, etc.
Greek letters are ($\Sigma, \Gamma,$ etc.) are usually used to denote alphabets.\\

\textit{Examples:}
\begin{itemize}
  \item $ \Sigma_1 = \{a,b,c\}$,
  \item $ \Sigma_2 = \{a,b,c,\dots,z\}$,
  \item $ \Gamma = \{a,b,c,\text{\$},\square\}$
\end{itemize}

A \textbf{word} over an alphabet $\Sigma$ is a string of finite length, 
in other words a finite sequence of symbols of $\Sigma$.\\

A (formal) \textbf{language} over an alphabet $\Sigma$ is defined as a 
set of words defined over $\Sigma$. The letters $w$, $u$ and $v$ are usually used to denote words.
The letters $s$, $t$ and $r$ are usually used for single letters. 
Languages are usually denoted by capital letters.\\

\textit{Examples of languages:}
\begin{itemize}
  \item $ L_1 = \{a,b,aab,abbcc\} $
  \item $ L_2 = \{acbb,accbb,acccbb,accccbb,\dots\}$
\end{itemize}
These languages are defined over the alphabet $\Sigma = \{a,b,c\}$.
\begin{itemize}
  \item $L_1$ is finite.
  \item $L_2$ is composed of all the words starting with the symbol $a$, 
  followed by at least one symbol $c$ and ending with $bb$.
\end{itemize}
From now on, only finite alphabets are considered. A language can be infinite, 
but each of its elements are always finite.

\subsection{Operations on Words}
The \textbf{concatenation} operation of two words consists of 
appending the second word to the end of the first one.
Concatenation of two words $ w_1 = s_1 s_2\dots s_m $ 
and $w_2 = t_1 t_2 \dots t_n$ is defined by
\begin{align*}
w_1 \cdot w_2 = s_1 \dots s_m t_1 \dots t_n.
\end{align*}
The concatenation operation is associative, but not commutative.\\

The operation that returns the \textbf{length}, i.e. 
the number of symbols of a word $w$ is denoted by $|w|$.\\

$\epsilon$ denotes a special word called the \textbf{empty string} 
(also denoted by $\lambda$). Some properties of $\epsilon$:
\begin{itemize}
  \item $|\epsilon| = 0$
  \item For any word $w$, we have $w \cdot \epsilon = \epsilon \cdot w = w$.
\end{itemize}

The \textbf{mirror} operation associates the reverse of a word $w = s_1 s_2 \dots s_n$,
which is denoted by $ w^R = s_n \dots s_2 s_1$.\\

Recursive definition:
\begin{enumerate}
  \item $\epsilon^R = \epsilon$
  \item $s^R = s$, for any symbol $s$
  \item $w^R = v^R \cdot u^R$, for any word $w =  u \cdot v$
\end{enumerate}

If $w = w^R$ then $w$ is called a 
\textbf{palindrome}, i.e. a word that reads the same forwards and backwards.\\

The \textbf{concatenation} of the word $w$ with itself $n$ times is denoted by $w^n$ where $n \geq 0$.\\

Recursive definition:
\begin{enumerate}
  \item $w^0 = \epsilon$
  \item $w^{n+1} = w^{n}w$
\end{enumerate}

The \textbf{set of all words} of length greater or equal to 1 
over an alphabet $\Sigma$ is denoted by $\Sigma^{+}$.\\

\textit{Example:}
\begin{align*}
  \Sigma^{+} = \{a,b,aa,ab,ba,bb,aaa,aab,\\aba,abb,baa,bab,\dots\} \text{, where } \Sigma = \{a,b\}
\end{align*}

The \textbf{set of all words} of length greater or equal to 
0 over an alphabet $\Sigma$ is denoted by $\Sigma^{*}$. Some remarks:
\begin{itemize}
  \item $\Sigma^{+}$ and $\Sigma^{*}$ are infinite if and only if $\Sigma \neq \emptyset$
  \item $\Sigma^{*} = \Sigma^{+} \cup \{\epsilon\}$
\end{itemize}

Definition of $\Sigma^{+}$:
\begin{enumerate}
  \item any element of $\Sigma$ belongs to $\Sigma^{+}$
  \item for all $a \in \Sigma$ and for all $w \in \Sigma^{+}$, we have $aw \in \Sigma^{+}$ 
\end{enumerate}

\subsection{Operations on Languages}
Operations on languages $L_1, L_2$ and $L$ are defined over an alphabet $\Sigma$:
\begin{itemize}
  \item \textbf{Union:} $L_1 \cup L_2$
  \item \textbf{Intersection:} $L_1 \cap L_2$
  \item \textbf{Complement:} $\overline{L} = \Sigma^{*} \backslash L$
\end{itemize}

Further operations include:
\begin{itemize}
  \item \textbf{Concatenation:} $L_1 \cdot L_2 = \{w_1 \cdot w_2 | w_1 \in L_1 $ and $w_2 \in L_2\} $
  \item \textbf{Neutral element:} $L_\epsilon = \{\epsilon\}$
  \item \textbf{Absorbance element:} $\emptyset$
\end{itemize}
For the neutral element holds
\begin{align*}
  L \cdot L_\epsilon = L_\epsilon \cdot L = L.
\end{align*}

For the absorbance element holds
\begin{align*}
  L \cdot \emptyset = \emptyset \cdot L = \emptyset.
\end{align*}

Note that $L_\epsilon \neq \emptyset$.\\

\textit{Example:} let $\Sigma = \{a,b\}, L_1 = \{a, b\}, L_2 = \{bac, b, a\} $.
\begin{align*}
  L_1 \cdot L_2 = \{abac, ab, aa, bbac, bb, ba\}
\end{align*}

The \textbf{concatenation} of the language $L$ with itself $n$ times is denoted by $L^n$ where $n \geq 0$.\\

Recursive definition:
\begin{itemize}
  \item $L^0 = L_\epsilon = \{\epsilon\}$
  \item $L^{n+1} = L^n \cdot L$
\end{itemize}

The \textbf{iterative closure} or \textbf{Kleene's closure} of a language $L$,
denoted as $L^{*}$, is the set of words resulting from the concatenation 
of a finite number of words of $L$. Formally
\begin{align*}
  L^{*} = L^0 \cup L^1 \cup L^2 \cup L^3 \cup \dots = \bigcup_{i \geq 0} L_i
\end{align*}

A variation of Kleene's closure called \textbf{Kleene plus} is defined as
\begin{align*}
  L^{+} =  L^1 \cup L^2 \cup L^3 \cup \dots = \bigcup_{i \geq 1} L_i \\ = L \cdot L^{*} = L^{*} \cdot L
\end{align*}

\textit{Example:} let $L = \{aa, bb, c\}$
\begin{align*}
  L^{*} = \{\epsilon, aa, bb, c, aaaa, aabb, aac, bbaa, bbbb,\\ bbc, caa, cbb, cc \dots\}
\end{align*}
